\documentclass{style/template}

\ResumeName{林继申}

\begin{document}

\ResumeContacts{
    (+86)151-4330-5542,
    \ResumeUrl{mailto:minmuslin@outlook.com}{minmuslin@outlook.com},
    \ResumeUrl{https://www.minmuslin.cn}{www.minmuslin.cn},
    \ResumeUrl{https://github.com/MinmusLin}{github.com/MinmusLin}
}

\ResumeTitle

\section{教育经历}

\ResumeItem{同济大学}
[\textnormal{计算机科学与技术学院|软件工程专业|}工学学士]
[2022.08-2026.06]
\begin{itemize}
    \item GPA:4.7/5.0 (专业前10\%),英语水平:CET-6,计算机专业课程成绩优异
    \item 获国际基因工程机器大赛(iGEM)国际金奖、中国国际大学生创新大赛(2024)上海市金奖(省级)和全国铜奖
\end{itemize}

\section{工作经历}

\ResumeItem{北京字节跳动科技有限公司}
[\textnormal{电商消费者业务货架导购部门|}后端开发工程师]
[2025.05至今]
\begin{itemize}
    \item \textbf{描述:}参与特卖营销业务,始终围绕业务价值与技术深度双主线推进:一方面通过需求落地直接提升业务指标(如ROI、转化率等),另一方面通过抽象通用解决方案,从频道需求中判断能够复用的能力,并沉淀成平台能力,为后续同类问题提供高效的解决路径。
    \item \textbf{超值购订阅站内信召回:}在交易链路,基于货补商品补贴价差引导用户订阅超值购频道,订阅后通过订阅卡/站内信渠道持续触达并召回用户,提升频道DAU和GMV。利用Golang的协程和Channel机制,对初版技术方案进行优化,设计内存占用较少的并发链路。内存占用率峰值变为优化前的2.85\%,全链路节省Hive数据总读取时间的97\%,日触达用户2000万。
    \item \textbf{超值购频道心智权益:}心智权益实际触达用户与实验分组不匹配,外加消息触达本身漏斗,因此难以做到显著,且现有短信需经浏览器中转才能唤起抖音APP,转化路径冗长,导致用户流失。通过将实验分流节点移至过滤逻辑后,确保入组用户均为有效触达目标,并采用Zlink动态参数拼接直接唤起抖音APP,消除依赖浏览器中转的不可控因素。实验分流节点后置有效解决过度曝光问题,提升实验显著性,并提升短信漏斗,30s主访用户占比与主动复访复购率正向显著。
\end{itemize}

\ResumeItem{太平洋保险集团长江养老保险股份有限公司}
[\textnormal{信息技术部门|}实习生]
[2024.07-2024.08]
\begin{itemize}
    \item \textbf{描述:}参与公司尽调助手开发与搭建,设计并实现尽调助手Workflow,涵盖问题分类、会议纪要生成、观点一致性分析及常规问题处理,通过自动化工作流程提高尽职调查效率,解决传统尽调过程中耗时长、成本高的问题。
    \item \textbf{成果:}成功部署并上线尽调助手,会议纪要生成与观点一致性分析功能显著提升效率,获得尽调团队积极反馈。
\end{itemize}

\section{项目经历}

\ResumeItem{智幕云:智慧幕墙数据集管理平台}
[项目负责人]
[2024.10-2025.03]
\begin{itemize}
    \item \textbf{描述:}本项目为同济大学智慧幕墙科研团队提供高效、安全的数据存储、处理和分析服务,推动土木工程科学与计算机科学的深度融合与创新应用。平台基于Spring Boot和Vue.js框架开发,基于GitHub CI/CD集成自动化测试与镜像构建,使用Docker Compose编排多容器服务,支持多维度数据集管理、权限控制及自动化运维,为智慧幕墙的仿真分析与实验研究提供核心数据支撑。
    \item \textbf{成果:}成功交付平台并稳定运行,日处理超50,000次数据请求,支撑金属幕墙锈蚀污损检测、玻璃幕墙爆裂检测、石材幕墙裂缝检测等十项核心研究。智慧幕墙科研平台被纳入同济大学重点科研基础设施。
\end{itemize}

\ResumeItem{\textbf{Atlas.Y}:用于优化酵母菌亚细胞定位的分子标签设计软件}
[软件开发组负责人]
[2024.03-2024.10]
\begin{itemize}
    \item \textbf{描述:}本软件为iGEM竞赛软件与人工智能赛道参赛项目,旨在结合人工智能技术解决合成生物学中的蛋白质定位问题。软件基于Spring Boot和Vue.js框架开发,接入AlphaFold和PyRosetta计算引擎,集成GNN模型,实现融合蛋白生成、蛋白质3D结构可视化、功能性和稳定性评估、定向进化、光遗传学定位等功能。
    \item \textbf{项目链接:}\ResumeUrl{https://2024.igem.wiki/tongji-software}{2024.igem.wiki/tongji-software}
    \item \textbf{成果:}本项目获得iGEM竞赛国际金奖(全球前10\%),得到近百位国际合成生物学工作者的高度认可,软件被麻省理工学院、清华大学等数个科研团队使用,同时我作为负责人之一带领同济团队前往法国巴黎线下展示项目。
\end{itemize}

\section{技术能力}

\begin{itemize}
    \item \textbf{工程能力:}熟练使用Git进行团队协作开发。熟练使用GitHub和GitLab的CI/CD进行自动化测试、构建和部署。熟练使用Docker和Docker Compose来管理容器、容器化运行服务。掌握软件测试的思想和一般性方法,能够使用JUnit、Vitest等测试框架。掌握软件性能优化的思想和一般性方法,能够使用火焰图等后端性能分析工具。掌握设计良好软件架构的经验和能力,对Golang应用的架构设计有充足经验。
    \item \textbf{后端开发:}熟悉Golang/Java语言和常用语言细节,有充足的Linux开发经验。熟悉KiteX/Hertz等RPC/HTTP框架,熟悉Golang生态。熟悉MySQL和Redis,具备数据库设计与优化能力。熟悉常用的设计模式,能够根据业务场景设计并实现高质量代码。
    \item \textbf{前端开发:}熟练掌握Web基本开发技能和组合式风格的Vue3和Typescript,熟悉Vue.js生态,有全栈开发经验。
    \item \textbf{软技能:}具备良好的问题解决、沟通协作及学习能力,能高效定位问题、协同团队开发,并持续学习行业新技术。
\end{itemize}

\end{document}
