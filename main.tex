\documentclass{style/template}

\ResumeName{林继申}

\begin{document}

\ResumeContacts{
    (+86)151-4330-5542,
    \ResumeUrl{mailto:minmuslin@outlook.com}{minmuslin@outlook.com},
    \ResumeUrl{https://minmuslin.cn}{minmuslin.cn},
    \ResumeUrl{https://github.com/MinmusLin}{github.com/MinmusLin}
}

\ResumeTitle

\section{教育经历}

\ResumeItem{同济大学}
[\textnormal{计算机科学与技术学院|软件工程专业|}工学学士]
[2022.08-2026.06]
\begin{itemize}
    \item GPA:4.7/5.0(专业前10\%),英语水平:CET-6,计算机专业课程成绩优异
    \item 获国际基因工程机器大赛(iGEM)国际金奖、中国国际大学生创新大赛(2024)上海市金奖(省级)和全国铜奖
\end{itemize}

\section{工作经历}

\ResumeItem{北京字节跳动科技有限公司}
[\textnormal{电商消费者业务货架导购部门|}后端开发工程师]
[2025.05至今]
\begin{itemize}
    \item \textbf{描述:}参与货架电商特卖营销业务,始终围绕业务价值与技术深度双主线推进:一方面通过需求落地直接提升业务指标(如GMV、DAU、ROI、转化率等),另一方面通过抽象通用解决方案,从频道需求中判断能够复用的能力,并沉淀成平台能力,为后续同类问题提供高效的解决路径。核心业务需求与技术需求产出如下:
    \item \textbf{超值购订阅站内信召回:}在交易链路,基于货补商品补贴价差引导用户订阅超值购频道,持续触达并召回用户。利用Golang协程和Channel机制,对初版技术方案进行优化,设计内存占用较少的并发链路。内存占用率峰值变为优化前的2.85\%,全链路节省Hive数据总读取时间的97\%,日触达用户2000万,提升频道DAU和GMV。
    \item \textbf{超值购频道心智权益:}发现并解决了心智权益实际触达用户与实验分组不匹配的过度曝光问题,通过重构实验分流逻辑(实验分流节点后置)和优化短信触达链路(采用Zlink直接唤起抖音APP),有效解决过度曝光问题,提升实验显著性和短信漏斗,30s主访用户占比与1日/7日主动复访复购率正向显著。
    \item \textbf{营销向实验过度曝光问题的通用解决方案:}频道级的通用解决方案,支撑秒杀、超值购等营销实验场景,提高研发效率。频道券补贴实验基于本方案进行优化,解决实验过度曝光问题,策略渗透效率提升20倍,为业务看清策略效果提供有效数据基础,在秒杀发券史上首次带动电商大盘人均支付GMV正向显著。
    \item \textbf{特卖钱效PSM-DID分析框架:}设计并实现特卖钱效PSM-DID分析框架,框架支持特卖营销业务(如夜市万人团、周末万人团、秒杀营销等玩法)的钱效分析。基于框架对夜市万人团业务进行钱效分析,反转实验呈现显著正向GMV增量,渗透用户的每日人均大盘GMV提升范围为4.54至4.83元,GMV-ROI在3.38至3.93之间。
\end{itemize}

\ResumeItem{太平洋保险集团长江养老保险股份有限公司}
[\textnormal{信息技术部门|}实习生]
[2024.07-2024.08]
\begin{itemize}
    \item \textbf{描述:}参与公司尽调助手开发与搭建,设计并实现尽调助手Workflow,涵盖问题分类、会议纪要生成、观点一致性分析及常规问题处理,通过自动化工作流程提高尽职调查效率,解决传统尽调过程中耗时长、成本高的问题。
    \item \textbf{成果:}成功部署并上线尽调助手,会议纪要生成与观点一致性分析功能显著提升尽调效率,获得尽调团队积极反馈。
\end{itemize}

\section{项目经历}

\ResumeItem{智幕云:智慧幕墙数据集管理平台}
[项目负责人]
[2024.10-2025.04]
\begin{itemize}
    \item \textbf{描述:}本项目为同济大学智慧幕墙科研团队提供高效安全的数据管理、存储、处理和分析服务,推动土木工程科学与计算机科学的深度融合与创新应用。平台基于Spring Boot和Vue.js框架开发,基于GitHub CI/CD集成自动化测试与镜像构建,使用Docker Compose编排多容器服务,为智慧幕墙仿真分析与实验研究提供核心数据支撑。
    \item \textbf{成果:}成功交付平台并稳定运行,日处理超50,000次数据请求,支撑金属幕墙锈蚀污损检测、玻璃幕墙爆裂检测、石材幕墙裂缝检测等共10项核心研究。智慧幕墙科研平台被纳入同济大学重点科研基础设施。
\end{itemize}

\ResumeItem{\textbf{Atlas.Y}:用于优化酵母菌亚细胞定位的分子标签设计软件}
[软件开发组负责人]
[2024.03-2024.10]
\begin{itemize}
    \item \textbf{描述:}本软件为iGEM竞赛软件与AI赛道参赛项目,旨在解决合成生物学中的蛋白质定位问题。软件基于Spring Boot和Vue.js框架开发,接入AlphaFold和PyRosetta计算引擎,集成GNN模型,实现融合蛋白生成、蛋白质3D结构可视化、功能性和稳定性评估、定向进化、光遗传学定位等功能。项目链接:\ResumeUrl{https://2024.igem.wiki/tongji-software}{2024.igem.wiki/tongji-software}
    \item \textbf{成果:}本项目获得iGEM竞赛国际金奖(全球前10\%),得到近百位国际合成生物学工作者的高度认可,软件被麻省理工学院、清华大学等数个科研团队使用,同时我作为负责人之一带领同济团队前往法国巴黎线下展示项目。
\end{itemize}

\section{技术能力}

\begin{itemize}
    \item \textbf{工程能力:}熟练使用Git进行团队协作开发。熟练使用GitHub和GitLab的CI/CD进行自动化测试、构建和部署。熟练使用Docker和Docker Compose来管理容器、容器化运行服务。掌握软件测试的思想和一般性方法,能够使用JUnit、Vitest等测试框架。掌握软件性能优化的思想和一般性方法,能够使用火焰图等后端性能分析工具。掌握设计良好软件架构的经验和能力,对Golang应用的架构设计有丰富经验。
    \item \textbf{后端开发:}开发不局限于特定编程语言和技术栈。熟练掌握Golang/Java语言,熟悉C/C++语言,有充足Linux开发经验。掌握微服务架构理念,熟悉KiteX/Hertz等RPC/HTTP框架。熟悉MySQL和Redis,具备数据库设计与优化能力。熟悉服务监控与告警体系,掌握常用设计模式,能够设计并实现高质量代码。
    \item \textbf{前端开发:}熟练掌握Web开发技能和组合式风格Vue3.js和TypeScript,熟悉Vue.js生态,有丰富全栈开发经验。
    \item \textbf{软技能:}具备良好的问题解决、沟通协作及学习能力,能高效定位问题、协同团队开发,并持续学习行业新技术。
\end{itemize}

\end{document}
